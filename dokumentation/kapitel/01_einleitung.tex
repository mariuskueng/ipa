%!TEX root = ../dokumentation.tex
\section{Projekorganisation}

\begin{figure}[!ht]
\begin{center}
\includegraphics[width=0.5\textwidth,angle=0]{./bilder/01_projektorganisation.pdf}
\caption[Projekorganisation]{Projekorganisation\footnotemark}
\end{center}
\end{figure}
\footnotetext{Eigene Darstellung}

\section{Vorkenntnisse}

Technologien
\begin{itemize}
    \item Python Grundkenntnisse
    \item Django Fortgeschrittene Kenntnisse
    \item Piston Grundkenntnisse
    \item jQuery Fortgeschrittene Kenntnisse
    \item HTML5 Gute Kenntnisse
    \item CSS3 Gute Kenntnisse
    \item AJAX Fortgeschrittene Kenntnisse\\
\end{itemize}

Anwendungen
\begin{itemize}
    \item Mehrere komplette Webauftritte realisiert
    \item Applikationen in Django erstellt(News, Blog, Produkteübersicht)
    \item Schnittstellen programmiert (XML, JSON)
    \item Per AJAX dynamische Inhalte laden und einfügen
    \item Dynamische Formulare abschicken und Request entgegenehmen
    \item DOM-Elemente manipulieren, per Events steuern
\end{itemize}
    
\section{Vorarbeiten}
Ich habe, um einen ersten Eindruck über die Funktionalität zu erhalten, den bestehenden yatplaner ausprobiert.
Ausserdem den allink.planer studiert in den, der Wochenplaner implementiert wird.
Ansonsten fand keine explizite Vorarbeit statt.

\section{Firmenstandards}
\begin{itemize}
    \item Betriebsystem: Mac OS X
    \item Editor: Textmate
    \item Entwicklungsumgebung: Python, Django (Python-Framework für Webapplikationen)
    \item Virtuelle Testumgebung: Django Serversimulation (per Terminal steuerbar)
    \item Deployment: per fabric-script auf Apache-Server mit WSGI-Protokoll
    \item Versionierung: Git, auf Github gehostet
    \begin{figure}[!ht]
    \begin{center}
    \includegraphics[width=0.99\textwidth,angle=0]{./bilder/git.png}
    \caption{A successful Git branching model von Vincent Driessen}
    \end{center}
    \end{figure}
\end{itemize}
