%!TEX root = ../dokumentation.tex
\section{Was ich gelernt habe}
In meiner 10-tägigen IPA habe ich folgendes gelernt:
\begin{itemize}
    \item wie man eine bestehende Applikation analysiert und Muss-/Kann-Ziele erfasst.
    \item planen wie viel Zeit man für eine Aufgabe benötigt
    \item Eine komplexe und datenlastige HTML-Ansicht umzusetzen
    \item Ziele zu definieren und für den zeitlichen Rahmen abzugrenzen
    \item verschachtelte Queries erstellt um eine kompakte Datenabfrage zu generieren
    \item eine dynamische Kommunikation mit dem Server herzustellen
    \item dynamische CRUD-Request erstellen
    \item die jQuery Library jQueryUI anzuwenden
    \item Testfälle erstellen und durchführen
\end{itemize}
\section{Positives}
Ich hatte sehr viel Spass an meiner Arbeit, da es sich um eine Webapplikation handelte die über AJAX CRUD-Funktionalität verfügt und ich etwas derartiges noch nie gemacht habe.
Was mich besonders überraschte war:
\begin{itemize}
    \item in Django Schnittstellen zu programmieren und diese zu erweitern ist relativ simpel.
    \item dass jQuery ohne grössere Probleme einen AJAX-Request bauen und per callback wieder Daten entgegen nehmen kann.
\end{itemize}
Was mich natürlich auch freute war, dass ich alle Ziele erreicht habe, wenn auch nicht unbedingt in der zu Beginn geplanten Umsetzungszeit.
\section{Negatives}
\subsection{Zeitplanung}
Dass ich die Zeitplanung nicht mit einer Timeline umgesetzt habe, machte mich nicht ganz zufrieden.
Jedoch empfand ich es als schwer ein so kleines Projekt so genau planen zu können.\\
Es traten in der Realität immer wieder kleine Probleme auf, die alle zusammen mein Zeitmanagement durcheinander gebracht hätten.
In Zukunft werde ich versuchen Aufgaben besser zu planen damit sie besser mit dem zeitlichen Rahmen übereinstimmen.
\subsection{Testing}
Im Nachhinein wurde mir klar, dass ich meine Testmethode besser von Anfang an in den Programmierprozess hätte einbeziehen sollen oder einen besseren Weg finden, um für diese Funktionalität ein Frontend-Testing durchzuführen.
Ich war der Ansicht, dass die Testmethode für die Testfälle ausreichend ist, da die Realisierung stark auf die Intuitiviät im Frontend ausgerichtet war.