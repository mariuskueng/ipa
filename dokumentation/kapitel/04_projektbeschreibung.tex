%!TEX root = ../dokumentation.tex
\section{Umfeld}
Momentan läuft der Wochenplaner vom allink.planer getrennt und kann nicht in den allink Projektablauf einbezogen werden.
Das Tool fungiert mehr als eine Tafel auf die Post-It's geklebt werden. D. h. die Aufgaben sind nicht direkt im Projektmanagementtool mit der Person verbunden.
Somit bietet das Tool keinen Mehrwert bei der Auswertung des Projektablaufs.\\
\\
Das Tool ist in Ruby on Rails programmiert, was nicht (mehr) dem Firmenstandard entspricht, und kann deshalb nur mühsam gewartet und erweitert werden.

\section{Präzisierung der Aufgabenstellung}
Da die Aufgabenstellung detailliert erfasst wurde und ein Prototyp existiert, ist relativ klar was die eigentliche Aufgabenstellung umfasst.
Zur weiteren Präzisierung, werde ich eine detaillierte IST-Analyse vornehmen. Diese wird mit dem Auftraggeber besprochen, 
damit die MUSS- und KANN-Ziele erfasst werden können und ersichtlich wird, welche Funktionalitäten überhaupt noch Sinn machen und was weggelassen werden kann.

\section{Analyse Wochenplaner (IST-Zustand)}

\subsection{Modellierung}
    Analysiert anhand der Rails-Modelle\\
    \\
    Person
    \begin{itemize}
        \item Name
        \item Locked Days (gesperrte Tage, Auswahl von Montag bis Freitag)
        \item Beschreibungstext für jeden gesperrten Tag
    \end{itemize}
    Task
    \begin{itemize}
        \item Name
        \item Datum (Fälligkeitsdatum, DATE)
        \item Person (Fremdschlüssel)
        \item Dauer (Stundenanzahl)
    \end{itemize}

\subsection{Darstellung}
Der Wochenplaner verfügt über mehrere Darstellungen:
\begin{itemize}
    \item Kalenderansicht der aktuellen Woche (Screenshot)
    \item Übersicht aller Angestellten
    \item Monatsansicht einer Person
    \item Übersicht aller erfassten Tasks
\end{itemize}

\subsection{Funktionsumfang}
\begin{itemize}
    \item Neue Task durch Doppelklick auf Personentag per Dialog erfassen (Screenshot)
    \item Klick auf Task öffnet die Detailansicht (um bspw. Änderungen vorzunehmen)
    \item Verschiebung einer Task per Drag \& Drop
    \begin{itemize}
        \item Task in anderen Tag verschieben
        \item Task einer anderen Person zuweisen
    \end{itemize}
    \item In der Kalenderwoche per Link vor- oder zurückspringen
    \item weitere 7 Tage in der Ansicht hinzufügen (kann auch rückgängig gemacht werden)
    \item Wochentage können für wiederkehrende Events (Schule, Frei) markiert werden
\end{itemize}
\subsection{Defizite}
    \begin{itemize}
        \item Die Sperrtage haben kein Startdatum und sind für jeden Wochentag erfasst
        \item Die Ajax-Funktionalität beschränkt sich auf Task erstellen und verschieben
        \item Die Bearbeitung der Tasks ist mühsam
    \end{itemize}

\section{Definition Ziele}
In einer Sitzung mit der Projektleitung wurde der IST-Zustand besprochen und die Ziele definiert.

\subsection{MUSS-Ziele}
\begin{table}[!ht]
\begin{center}
    \begin{tabular}{llp{8cm}l}
        \toprule Nr & Funktion & Beschreibung \\
        \midrule 1 & Task erfassen & Der Benutzer kann eine neue Task erfassen. \\
        \midrule 2 & Task Namen geben & Der Benutzer kann einem Task einen Namen geben. \\
        \midrule 3 & Task Datum geben & Der Benutzer kann einem Task ein Datum geben. \\
        \midrule 4 & Task Person zuweisen & Der Benutzer kann einem Task eine Person zuweisen. \\
        \midrule 5 & Task Dauer geben & Der Benutzer kann einem Task eine Dauer (in Stunden) geben. \\
        \midrule 6 & Task bearbeiten & Der Benutzer kann einen Task bearbeiten. \\ 
        \midrule 7 & Task löschen & Der Benutzer kann einen Task löschen. \\
        \midrule 8 & Person hinzufügen & Der Benutzer kann eine Person dem Wochenplaner hinzufügen. \\
        \midrule 9 & Sperrtag Person zuweisen & Der Benutzer kann einer Person einen Sperrtag zuweisen. \\
        \midrule 10 & Sperrtag Namen geben & Der Benutzer kann einem Sperrtag einen Namen geben. \\
        \midrule 11 & Sperrtag bearbeiten & Der Benutzer kann einen Sperrtag bearbeiten.\\
        \midrule 12 & Sperrtag löschen & Der Benutzer kann einen Sperrtag löschen.\\
        \midrule 13 & Task einem Projekt zuweisen & Der Benutzer kann einen Task einem Projekt zuweisen.\\
        \midrule 14 & Person zu Wochenplaner & Eine Person zum Wochenplaner hinzufügen.\\
        \midrule 15 & Person als Partner & Eine Person als Partner kennzeichnen.\\
        \bottomrule
    \end{tabular}
    \caption{Zwingend umzusetzende Funktionen des Prototypen}
    \label{tab:muss_funktionen}
\end{center}
\end{table}
\subsection{KANN-Ziele}
\begin{table}[!ht]
\begin{center}
    \begin{tabular}{llp{8cm}l}
        \toprule Nr & Funktion & Beschreibung & Priorität \\
        \midrule 16 & Task duplizieren & Der Benutzer kann einen Task duplizieren. & 1\\
        \midrule 17 & Sperrtag Startdatum geben & Der Benutzer kann einem Sperrtag ein Startdatum geben. &  3\\ 
        \midrule 18 & Warnmeldung Max & Es erscheint eine Warnmeldung wenn die Summe aller Arbeitsstunden pro Tag 8h überschreiten. & 4\\
        \midrule 19 & Tasks sortieren & Tasks können innerhalb eines Tages sortiert werden. & 2\\
        \bottomrule
    \end{tabular}
    \caption{Nicht zwingend umzusetzende Funktionen des Prototypen}
    \label{tab:kann_funktionen}
\end{center}
\end{table}

% \section{Lösungsvarianten (SOLL-Möglichkeiten)}
% 
% \section{Entscheidung Projektleitung Variante}