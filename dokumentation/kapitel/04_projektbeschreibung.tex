%!TEX root = ../dokumentation.tex
\section{Umfeld}
Momentan läuft der Wochenplaner vom allink.planer getrennt und kann nicht in den allink Projektablauf einbezogen werden.
Das Tool fungiert mehr als eine Tafel auf die Post-It's geklebt werden. D. h. die Aufgaben sind nicht direkt im Projektmanagenttool mit der Person verbindet.
Somit bietet das Tool keinen Mehrwert bei der Auswertung des Projektablaufs.\\
\\
Das Tool ist in Ruby on Rails programmiert, was nicht (mehr) dem Firmenstandard entspricht, und kann deshalb nur mühsam gewartet und erweitert werden.

\section{Analyse Wochenplaner (IST-Zustand)}
\subsection{Ansichten}
Der Wochenplaner verfügt über mehrere Ansichten:
\begin{itemize}
    \item Kalenderansicht der aktuellen Woche (Screenshot?)
    \item Übersicht aller Angestellten
    \item Monatsansicht einer Person
    \item Übersicht aller erfassten Tasks
\end{itemize}

\subsection{Funktionsumfang}
\begin{itemize}
    \item Neue Task durch Doppelklick auf Personentag per Dialog erfassen
    \item Doppelklick auf Task öffnet die Detailansicht (um bspw. Änderungen vorzunehmen)
    \item Verschiebung einer Task per Drag \& Drop
    \begin{itemize}
        \item Task auf in anderen Tag verschieben
        \item Task einer anderen Person zuweisen
        \item Task innerhalb eines Tages verschieben
    \end{itemize}
    \item In der Kalenderwoche per Link vor- oder zurückspringen
    \item weitere 7 Tage in der Ansicht hinzufügen (kann auch rückgängig gemacht werden)
    \item Wochentage können für wiederkehrende Events (Schule, Frei) markiert werden
\end{itemize}
\section{Definition Muss / Kann Ziele}
Sitzung Muss Kann Ziele\\
\\
MUSS\\
- Ansichten
    - Ansicht der aktuellen Woche (Startseite)
    - Alle anderen Übersichten können in den Django-admin ausgelagert werden


- Bedienung
    - Doppelklick auf Personentag öffnet Add Task Dialog
        - abschicken des Forms speichert die Task
    - Doppelklick auf Task öffnet Task Detailansicht
    - Verschiebung einer Task per Drag & Drop
        - Task in einen anderen Tag verschieben
        - Tasks innerhalb eines Tages sortieren

\\
KANN\\
- Doppelklick auf Task öffnet Dialog & Task kann direkt bearbeitet und abgespeichert werden
- Per Tastenkombination kann eine Task kopiert werden und neu eingefügt werden
- Löschen per delete taste

\section{Lösungsvarianten (SOLL-Möglichkeiten)}

\section{Entscheidung Projektleitung Variante}
