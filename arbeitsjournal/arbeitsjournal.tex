%!TEX root = ../dokumentation/dokumentation.tex
\section{13.03.2012}
Beginn IPA yatplaner
\begin{itemize}
    \item Dokumentation auf github hosten und versionieren
    \item Kapitelstrukur aufbauen 
    Teil 1\\
    \item Zeitplan erstellen (Planung)
    \item Aufgabenstellung erfassen
    \item Projektorganisation grafisch erfassen
    \item Designbesprechung mit Dave
    \item Vorkenntnisse erfassen
    \item Vorarbeiten erfassen
    \item Firmenstandards erfassen
    \item Teil 1 Der Doku erfasst
    Teil 2\\
    \item Umfeld erfassen
    \item IST-Analyse vorerfassen (Ansichten, Funktionalität)
\end{itemize}
\section{14.03.2012}
\begin{itemize}
    %Teil 2\\
    \item IST-Analyse vorerfassen (Modellierung der Rails App)
    \item Sitzungen mit Silvan Spross und Marc Egli zur Definition der MUSS- und KANN-Ziele
    \item MUSS-Ziele erfasst
    \item KANN-Ziele erfasst
    \item Planung abgeschlossen
    \item ERM für Wochenplaner erstellt
    \item allink.planer github repository gecloned
    \item Entwicklungsumgebung aufgesetzt
    \item mit Django-Modellierung begonnen (Task, DayConfiguration)
\end{itemize}
\section{15.03.2012}
\begin{itemize}
    \item Kapitel Realisierung weiterführen
    \item Django admin einrichten für week
    \item Testdaten (Personen) in die Entwicklungsumgebung importieren
    \item Problem: Die Wochentage zusammenzufassen war relativ knifflig damit man durch jeden Tag iterieren kann
    \item View erstellen für alle Tasks, Personen und Tage
    \item Template erstellt und Tabelle so aufgesetzt damit Daten korrekt eingefügt werden
    \item Template begonnen zu stylen
\end{itemize}