%!TEX root = ../dokumentation/dokumentation.tex
\section{13.03.2012}
Beginn IPA yatplaner
\begin{itemize}
    \item Dokumentation auf github hosten und versionieren
    \item Kapitelstrukur aufbauen 
    Teil 1\\
    \item Zeitplan erstellen (Planung)
    \item Aufgabenstellung erfassen
    \item Projektorganisation grafisch erfassen
    \item Designbesprechung mit Dave
    \item Vorkenntnisse erfassen
    \item Vorarbeiten erfassen
    \item Firmenstandards erfassen
    \item Ziel erreicht: Teil 1 Der Doku erfasst
    Teil 2\\
    \item Umfeld erfassen
    \item IST-Analyse vorerfassen (Ansichten, Funktionalität)
\end{itemize}
\section{14.03.2012}
\begin{itemize}
    %Teil 2\\
    \item IST-Analyse vorerfassen (Modellierung der Rails App)
    \item Sitzungen mit Silvan Spross und Marc Egli zur Definition der MUSS- und KANN-Ziele
    \item MUSS-Ziele erfasst
    \item KANN-Ziele erfasst
    \item Planung abgeschlossen
    \item ERM für Wochenplaner erstellt
    \item allink.planer github repository gecloned
    \item Entwicklungsumgebung aufgesetzt
    \item Ziel erreicht: Teil 2 zum Grossteil der Dokumentation erfasst und mit Realisierung begonnen
    \item mit Django-Modellierung begonnen (Task, DayConfiguration)
    \item Habe bis jetzt für Doku mehr Zeit gebraucht als im Zeitplan vorgesehen
\end{itemize}
\section{15.03.2012}
\begin{itemize}
    \item Kapitel Realisierung weiterführen
    \item Django admin einrichten für week
    \item Testdaten (Personen) in die Entwicklungsumgebung importieren
    \item Problem: Die Wochentage zusammenzufassen war relativ knifflig damit man durch jeden Tag iterieren kann
    \item View erstellen für alle Tasks, Personen und Tage
    \item Template erstellt und Tabelle so aufgesetzt damit Daten korrekt eingefügt werden
    \item Template begonnen zu stylen
    \item Ziel erreicht: Daten korrekt aus Modell laden und in Template einsetzen, Template funktioniert
    \item (Bin im Zeitplan mit Realisierung)
\end{itemize}
\section{16.03.2012}
    \begin{itemize}
        \item Ansicht funktioniert für erste Versuche mit jQueryUI
        \item Task sind verschiebbar und sortierbar in andere Tage (ohne callback)
        \item alle Tasks werden zu javascript Objekte der Klasse Task
        \item jquery methoden hinzugefügt
        \item dialog hinzugefügt
        \item Task Objekt erweitert mit save methode wenn eine Task gespeichert wird
        \item Django Task Formular
        \item Django Task Piston Handler (json)
        \item AJAX POST schicken
        \item json piston nimmt POST entgegen und führt das formular aus welches die daten in die db speicher
        \item wenn neuer eintrag gemacht wurde wird die neue task.id zurückgeschickt um es lokal im js task objekt zu speichern
        \item Ziel: jQueryUI einrichten, mind. POST per Ajax möglich und dynamisches einfügen in Tabelle (beide erreicht)
    \end{itemize}
\section{19.03.2012}
    \begin{itemize}
        \item Todo: Taskobjekte holen der aktuellen woche
        \item Hilfestellung Silvan: PUT Funktionalität in Django abhandeln bei existierendem Objekt
        \item Verschieben von Task (PUT Funktionalität)
        \item Klick auf eine Task öffnet Dialog und man kann die Task bearbeiten und abspeichern
        \item Bugfix: wenn man eine Task draggt öffnet es danach den Bearbeiten-Dialog. Per deaktivieren des click events kann dies behoben werden.
        \item Bugfix: man konnte eine neu erstellte Task nicht verschieben
        \item Projekte werden dynamisch in den Dialog eingefügt und können ausgewählt werden
        \item Ziel: PUT Funktionalität erstellen für Task verschieben und ändern
        \item Dokumentation weiter führen
        \item Besprechung mit Silvan Stand IPA
    \end{itemize}
\section{20.03.2012}
    \begin{itemize}
        \item Dokumentation weiter geschrieben
        \item feature task duplizieren ausprobiert (Kann-Ziel)
        \item Bugfix: Taskzelle ist zu wenig hoch um problemlos neue Task in Zelle zu verschieben
        \item Bugfix: Bei Bearbeitung einer Task wird aktuelles Projekt nicht mitgeschickt
    \end{itemize}
\section{21.03.2012}
    \begin{itemize}
        \item Sperrtag bild ausgewechselt
        \item Bugfix: wenn man im Template die Woche wechselt kann das Datumsformat nicht vearbeitet werden
        \item Queries optimieren mit Marc, verschachtelte Tuples für queries
        \item Dokumentation weiter geschrieben
        \item Kleinere Optimierungen
        \item Cloning weiter versuchen
    \end{itemize}
