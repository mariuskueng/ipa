%  --------------------------------------------------------------------------
%  IPA yatplaner Dokumentation
%  Created by Marius Küng on 2012-05-06.
%  --------------------------------------------------------------------------

%  --------------------------------------------------------------------------
%  Latex Document Settings
%  --------------------------------------------------------------------------

\documentclass[
11pt, % Schriftgrösse
a4paper, % A4 Papier
BCOR10mm, % Absoluter Wert der Bindekorrektur, z.B. BCOR1cm
DIV14, % Satzspiegel festlegen siehe
       % http://www.ctex.org/documents/packages/nonstd/koma-script.pdf
footsepline = false, % Trennlinie zwischen Textkörper und Fußzeile
                     % bei normalen Seiten
headsepline, % Trennlinie zwischen Kopfzeile und Textkörper
             % bei normalen Seiten
oneside, % Zweiseitig
openright,
halfparskip, % Europäischer Satz mit Abstand zwischen den Absätzen
abstracton, % inkl. Abstract
listof=totocnumbered, % Abb.- und Tab.verzeichnis im Inhaltsverzeichnis
bibliography=totocnumbered % Lit.zeichnis in Inhaltsverzeichnis aufnehmen
]{scrreprt}

\usepackage[automark]{scrpage2} % Gestaltung von kopf- und Fußzeile
\usepackage[ngerman]{babel}
\usepackage[ngerman]{translator}
\usepackage{tocbasic}
\usepackage[utf8]{inputenc}
\usepackage{lmodern} % Latin Modern
\usepackage[T1]{fontenc}
\usepackage{hyphenat}
\usepackage{ae} % Schöne Schriften für PDF-Dateien

% Tradmark
\def\TTra{\textsuperscript{\texttrademark}}

%1.5 Zeilenabstand
\usepackage[onehalfspacing]{setspace}

% Festlegung des Seitenstils (scrpage2)
\pagestyle{scrheadings}
\clearscrheadfoot
\automark[chapter]{section}

% \lehead{\sffamily\upshape\headmark}
% \cehead{}
% \rehead{}
% \lefoot[\pagemark]{\upshape \pagemark}
% \cefoot{}
% \refoot{}
% \lohead{}
% \cohead{}
\lohead{\sffamily\upshape\headmark}
\lofoot{}
\cofoot[\today]{}
\rofoot[\pagemark]{\scshape \pagemark}

% Surround parts of graphics with box
\usepackage{boxedminipage}

% Package for including code in the document
\usepackage{listings}

% If you want to generate a toc for each chapter (use with book)
\usepackage{minitoc}
\usepackage{longtable}

% Abkürzungsverzeichnis erstellen.
\usepackage[printonlyused]{acronym}

% schöne Tabelle zeichnen
\usepackage{booktabs}
\renewcommand{\arraystretch}{1.4} %Die Zeilenabstände in Tabellen angepasst.

% für variable Breiten
\usepackage{tabularx}

% Durchgestrichener Text
\usepackage[normalem]{ulem} %emphasize weiterhin kursiv

% This is now the recommended way for checking for PDFLaTeX:
\usepackage{ifpdf}

\usepackage{eurosym}

\usepackage{natbib}

\usepackage{paralist}

\usepackage{array,ragged2e}

\usepackage[hyperfootnotes=false]{hyperref}
\hypersetup{
  bookmarks=true,         % show bookmarks bar?
  unicode=true,           % non-Latin characters in Acrobat’s bookmarks
  pdftoolbar=true,        % show Acrobat’s toolbar?
  pdfmenubar=true,        % show Acrobat’s menu?
  pdffitwindow=true,      % window fit to page when opened
  pdfstartview={FitH},    % fits the width of the page to the window
  pdftitle={IPA Dokumentation},   
  pdfauthor={Marius Küng},
  pdfsubject={Dokumentation IPA yatplaner},
  pdfcreator={TeX Live 2009},
  pdfproducer={pdfTeX, Version 3.1415926-1.40.10},
  pdfnewwindow=true,      % links in new window
  colorlinks=true,       % false: boxed links; true: colored links
  linkcolor=blue,          % color of internal links
  citecolor=black,        % color of links to bibliography
  filecolor=magenta,      % color of file links
  urlcolor=cyan          % color of external links
  % linkcolor=black,          % color of internal links
  % citecolor=black,        % color of links to bibliography
  % filecolor=black,      % color of file links
  % urlcolor=black          % color of external links
}

\ifpdf
    \usepackage[pdftex]{graphicx}
\else
    \usepackage{graphicx}
\fi

\makeatletter 
\let\orgdescriptionlabel\descriptionlabel 
\renewcommand*{\descriptionlabel}[1]{% 
  \let\orglabel\label 
  \let\label\@gobble 
  \phantomsection 
  \edef\@currentlabel{#1}% 
  %\edef\@currentlabelname{#1}% 
  \let\label\orglabel 
  \orgdescriptionlabel{#1}% 
} 
\makeatother 

%  --------------------------------------------------------------------------
%  Start Document
%  --------------------------------------------------------------------------
\title{Dokumentation IPA yatplaner}

\author{IPA in Applikationsentwicklung\\
    \\
    Auszubildender - Marius Kümg\\
	Auftraggeber - allink.creative GmbH\\
    Projektleiter - Silvan Spross\\
    Experte - Antonio Di Luzio\\
    Durchführungsort - allink.creative GmbH\\
	\\
	Informatikmittelschule Basel}

\date{März 2011 bis Juni 2011}

\begin{document}
    \ifpdf
        \DeclareGraphicsExtensions{.pdf, .jpg, .tif}
    \else
        \DeclareGraphicsExtensions{.eps, .jpg}
    \fi

    \pagenumbering{Alph}
  
    \maketitle
    
    

    \pagenumbering{arabic}
  
    \tableofcontents
    
    \part{Umfeld und Ablauf}
        \chapter{Einführung}
            Die nachfolgende Grafik zeigt den neuen Projektablauf 
mit einer Aufteilung in vier Hauptabschnitte. Jedem Abschnitt wurden Resultate
hinzugefügt, die während dieser Phase erarbeitet oder durchgeführt werden sollen.

% \begin{figure}[htbp]
% \begin{center}
% %\includegraphics[width=0.99\textwidth,angle=0]{./bilder/test.pdf}
% \caption[Aufteilung des Projektablaufes mit Resultaten]{Aufteilung des Projektablaufes mit Resultaten\footnotemark}
% \label{pic:01_projektablauf}
% \end{center}
% \end{figure}
% \footnotetext{Eigene Darstellung}

\begin{center}
  \begin{tabular}{ l | c | r | }
    \hline
    1 & 2 & 3 \\ \hline
    4 & 5 & 6 \\ \hline
    7 & 8 & 9 \\
    \hline
  \end{tabular}
\end{center}
        \chapter{Augabenstellung}
            \section{Titel der Facharbeit} 
Webapplikation zur Ressourcenplanung von allink.creative
    
\section{Thematik}
Es soll eine Webapplikation mit Django erstellt werden, mit welcher die Geschäftsleitung die Ressourcenplanung der Mitarbeiter vornehmen kann. 
Damit soll eine ältere Webapplikation abgelöst werden.
\section{Klassierung}
    
\begin{itemize}
    \item Applikationsentwicklung OO
    \item UNIX / Linux
    \item andere\_Programmiersprache
\end{itemize}
    
\section{Durchführungsblock}
Startblock 1: 12.03.2012 - 23.04.2012\\
IPA-Durchführung: 12.03.2012 - 23.04.2012\\
Einreichung bis: Montag, 30.01.2012\\
    
\section{Ausgangslage}
Bei allink besteht seit Mitte 2010 ein rudimentäres Ressourcenplanungstool. Zur Entwicklung wurden damals jedoch Technologien verwendet, die heute nicht mehr zur Kernkompetenz von allink zählen. Da dieses Tool jedoch jeden Freitag zur Planung der nächsten Woche verwendet wird, ist es seit längerem überfällig es in die bestehende Managementapplikation zu integrieren.

Dank der übermässig langen Testphase des Prototypen sind nun die Anforderungen an das definitive Tool gut bekannt. Daher soll eine Webapplikation mit Django erstellt werden, mit welcher die Geschäftsleitung von allink.creative die Ressourcenplanung der Mitarbeiter vornehmen kann. Damit soll eine ältere Webapplikation abgelöst werden.
    
\section{Detaillierte Aufgabenstellung}
    
Das bestehende Tool namens \"Yatplaner\" soll als Modul im bestehenden Management Tool namens \"allink.planer\" reimplementiert werden. Dabei soll die Bedienbarkeit verbessert werden. Das Ziel ist es das neue Tool so intuitiv bedienen zu können, dass für die Geschäftsleitung keine Schulung nötig ist. Der Praxistest wird voraussichtlich in der letzten IPA Woche an der Wochenplansitzung durchgeführt.

Nebst der begleitenden IPA Dokumentation, wo unter anderem der Funktionsumfang des bestehenden Tools analysiert wird, wird keine zusätzliche Dokumentation gefordert. Der Funktionsumfang des bestehenden Tools soll vom Lernenden in einer Analysephase aufgenommen werden. Dabei sollen die bestehenden Features als Muss- und mögliche neue Features als Kann-Ziele ausformuliert werden.

Der Quellcode des bestehenden Tools ist unter folgender Adresse einsehbar: \\https://github.com/sspross/yatplaner/tree/rails

\section{Mittel und Methoden}
Folgende Technologien sind zwingend zu verwenden:

\begin{itemize}
    \item  Python 2.6
    \item  Django 1.3
    \item  Piston 2.3
    \item  jQuery 1.8
    \item  HTML5
    \item  CSS3
\end{itemize}

Das Tool soll in folgenden Browsern fehlerfrei funktionieren:

\begin{itemize}
    \item Firefox >= 8
    \item Chrome >= 10
    \item Safari >= 5
\end{itemize}

Der Internet Explorer muss explizit nicht unterstützt werden.
    
\section{Vorkenntnisse}
Dem Lernenden sind alle genannten Technologien bereits bekannt. Seit Beginn seines Praktikums im August 2011 setzt er sich damit auseinander. Gewisse Kombinationen wie z.B. mit jQuery einen AJAX Request zu erstellen, sind jedoch Neuland. 
    
\section{Vorarbeiten}
Es findet keine explizite Vorarbeit statt. 
    
\section{Neue Lerninhalte}
Wie bereits erwähnt sind dem Lernenden alle Technologien bereits bekannt. Jedoch sind gewisse Kombinationen noch nie vom Lernenden selbst angewandt worden. Das Know-how ist bei allink ausreichend vorhanden. Der Lernende kann zudem auf eine Vielzahl von bestehenden Projekten zurückgreifen, wo er unzählige Beispiele studieren kann. 
    
\section{Arbeiten in den letzten 6 Monaten}
Der Lernende hat überwiegend Webseiten mit den oben genannten Technologien umgesetzt. Zu seinen umfassendsten Arbeiten zählen bis jetzt eine Webseite eines Immobilienunternehmens und einer Eventplattform eines Finanzkonzerns. Für ersteres arbeitete der Lernende rund 250 Stunden daran. 

    \part{Projekt}
        \chapter{Analyse}
            {\textbf{IST/SOLL Analyse yatplaner.orwell.ch}}

Bestehende Features:
\begin{itemize}
    \item Ansichten
    \begin{itemize}
        \item Wochenansicht (wie bisher)
        \item Personenübersicht
        \item Monatsübersicht pro Person
    \end{itemize}
    \item Bedienung
    \begin{itemize}
        \item Doppelklick auf Personentag öffnet Add Task Dialog
        \item Doppelklick auf Task öffnet Task Detailansicht
        \item Verschiebung einer Task per Drag \& Drop
    \end{itemize}
\end{itemize}

\begin{table}[!th]
\begin{tabular}{|l|c|r|}
\hline
first  &  row  &  data \\
second &  row  &  data \\
\hline
\end{tabular}
\caption{This is the caption}
%\label{ex:table}
\end{table}
%The table is numbered \ref{ex:table}.
    
    \part{Arbeitsjournal}
        %!TEX root = ../dokumentation/dokumentation.tex
\section{13.03.2012}
Beginn IPA yatplaner
\begin{itemize}
    \item Dokumentation auf github hosten und versionieren
    \item Kapitelstrukur aufbauen 
    \item Zeitplan erstellen (Planung)
    \item Aufgabenstellung erfassen
    \item Projektorganisation grafisch erfassen
    \item Designbesprechung mit Dave
    \item Vorkenntnisse erfassen
    \item Vorarbeiten erfassen
    \item Firmenstandards erfassen
    \item Ziel erreicht: Teil 1 Der Doku erfasst
    \item Umfeld erfassen
    \item IST-Analyse vorerfassen (Ansichten, Funktionalität)
\end{itemize}
\section{14.03.2012}
\begin{itemize}
    %Teil 2\\
    \item IST-Analyse vorerfassen (Modellierung der Rails App)
    \item Sitzungen mit Silvan Spross und Marc Egli zur Definition der MUSS- und KANN-Ziele
    \item MUSS-Ziele erfasst
    \item KANN-Ziele erfasst
    \item Planung abgeschlossen
    \item ERM für Wochenplaner erstellt
    \item allink.planer github repository gecloned
    \item Entwicklungsumgebung aufgesetzt
    \item Ziel erreicht: Anfang Teil 2 der Dokumentation erfasst und mit Realisierung begonnen
    \item mit Django-Modellierung begonnen (Task, DayConfiguration)
    %\item Habe bis jetzt für Doku mehr Zeit gebraucht als im Zeitplan vorgesehen
\end{itemize}
\section{15.03.2012}
    \begin{itemize}
        \item Kapitel Realisierung weiterführen
        \item Django admin einrichten für week
        \item Testdaten (Personen) in die Entwicklungsumgebung importieren
        \item Problem: Die Wochentage zusammenzufassen war relativ knifflig damit man durch jeden Tag iterieren kann
        \item View erstellen für alle Tasks, Personen und Tage
        \item Template erstellt und Tabelle so aufgesetzt damit Daten korrekt eingefügt werden
        \item Template angefangen zu stylen (CSS)
        \item Ziel erreicht: Daten korrekt aus Modell laden und in Template einsetzen, Template funktioniert
        \item (Bin im Zeitplan mit Realisierung)
    \end{itemize}
\section{16.03.2012}
    \begin{itemize}
        \item Ansicht funktioniert für erste Versuche mit jQueryUI
        \item Task sind verschiebbar und sortierbar in andere Tage (ohne callback/Änderungen werden nicht gespeichert)
        \item alle Tasks werden zu javascript Objekten der Klasse Task
        \item jQuery Methoden hinzugefügt (Click-Events)
        \item Task-Dialog hinzugefügt um neue Task zu erfassen
        \item Task Objekt erweitert mit save Methode wenn eine Task gespeichert wird
        \item Django Task Formular
        \item Django Task Piston Handler (json)
        \item AJAX POST schicken
        \item piston nimmt json-POST entgegen und führt das Formular, aus welches die Daten in die Datenbank speichert
        \item wenn eine neue Task erstellt wurde, wird die neue task.id zurückgeschickt um das neue Javascript Task Objekt abzuspeichern
        \item Ziel erreicht: jQueryUI einrichten, mind. POST per Ajax möglich und dynamisches einfügen in Tabelle
    \end{itemize}
\section{19.03.2012}
    \begin{itemize}
        \item Taskobjekte der aktuellen Woche in javascript dictionary speichern
        \item Hilfestellung Silvan: PUT Funktionalität in Django abhandeln bei existierendem Objekt
        \item Verschieben von Task (PUT Funktionalität)
        \item Klick auf eine Task öffnet Dialog und man kann die Task bearbeiten und abspeichern
        \item Bugfix: wenn man eine Task verschoben wird, öffnet es danach den Bearbeiten-Dialog. Per deaktivieren des click events kann dies behoben werden.
        \item Bugfix: man konnte eine neu erstellte Task nicht verschieben
        \item Projekte werden dynamisch in den Dialog eingefügt und können ausgewählt werden
        \item Task einem Projekt zuweisen
        \item Ziel: PUT Funktionalität erstellen für Task verschieben und ändern
        \item Dokumentation weiter führen
        \item Besprechung mit Silvan Stand IPA
    \end{itemize}
\section{20.03.2012}
    \begin{itemize}
        \item Umsetzung MUSS-Ziele erreicht
        \item Dokumentation weiter geschrieben
        \item feature task duplizieren ausprobiert (Kann-Ziel)
        \item Bugfix: Taskzelle ist zu wenig hoch um problemlos neue Task in Zelle zu verschieben
        \item Bugfix: Bei Bearbeitung einer Task wird aktuelles Projekt nicht mitgeschickt
    \end{itemize}
\section{21.03.2012}
    \begin{itemize}
        \item Sperrtag bild ausgewechselt
        \item Bugfix: wenn man im Template die Woche wechselt kann das Datumsformat nicht vearbeitet werden
        \item Queries optimieren mit Marc, verschachtelte Tuples für queries
        \item Dokumentation weiter geschrieben
        \item Kleinere Optimierungen
        \item Cloning weiter versuchen
    \end{itemize}
\section{22.03.2012}
    \begin{itemize}
        \item Fehlerbehebungen im Template: Kalenderwoche wird nicht aktualisiert
        \item KANN-Ziel Tasks sortieren
        \item KANN-Ziel Warnung wenn Tasks in Tag mehr als 8h Stunden betragen
        \item KANN-Ziel Einer Tageskonfiguration ein Startdatum setzen
        \item Bugfix wenn neue Task erstellt wurde konnte sie nicht direkt bearbeitet werden.
        \item Besprechung Sperrtage habe ein Start- und Enddatum und können von weiteren Sperrtagen ersetzt ohne aus der DB gelöscht zu werden
        \item Mit Unterstützung von Marc die beste Lösung finden die Sperrtage auszulesen (Bei jedem Tag abfragen)
        \item Bugfix man kann nicht weniger als 1 Woche anzeigen
        \item Bugfix wenn man die Tagesanzahl erhöht wird aktuelles Datum nicht berücksichtigt
        \item Besprechung wie das Task-cloning am besten implementiert wird
    \end{itemize}
\section{23.03.2012}
    \begin{itemize}
        \item Task kann per Klick dupliziert werden (es wird eine neue mit gleichem Inhalt erzeugt und unten eingefügt)
        \item Task kann per Klick gelöscht werden
        \item Doku schreiben
        \item Mit Silvan Wochenplaner angeschaut und Darstellungsanpassungen vorgenommen
        \item Abnahme Wochenplaner Silvan
    \end{itemize}
\section{26.03.2012}
    \begin{itemize}
        \item Bugfix bei verschieben von Task auf max. Stunden Anzahl prüfen.
        \item Bugfix wenn bei neuem Task keine Stunden angegeben sind.
        \item Dokumentation überarbeiten
        \item Dokumentation fertigstellen
        \item IPA Dokumentation einreichen
    \end{itemize}
    
    \listoffigures
    \listoftables
    % \lstlistoflistings
  
    \bibliographystyle{unsrtnat}
    \bibliography{literaturverzeichnis}
        
\end{document}
