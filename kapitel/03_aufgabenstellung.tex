\section{Titel der Facharbeit} 
Webapplikation zur Ressourcenplanung von allink.creative
    
\section{Thematik}
Es soll eine Webapplikation mit Django erstellt werden, mit welcher die Geschäftsleitung die Ressourcenplanung der Mitarbeiter vornehmen kann. 
Damit soll eine ältere Webapplikation abgelöst werden.
\section{Klassierung}
    
\begin{itemize}
    \item Applikationsentwicklung OO
    \item UNIX / Linux
    \item andere\_Programmiersprache
\end{itemize}
    
\section{Durchführungsblock}
Startblock 1: 12.03.2012 - 23.04.2012\\
IPA-Durchführung: 12.03.2012 - 23.04.2012\\
Einreichung bis: Montag, 30.01.2012\\
    
\section{Ausgangslage}
Bei allink besteht seit Mitte 2010 ein rudimentäres Ressourcenplanungstool. Zur Entwicklung wurden damals jedoch Technologien verwendet, die heute nicht mehr zur Kernkompetenz von allink zählen. Da dieses Tool jedoch jeden Freitag zur Planung der nächsten Woche verwendet wird, ist es seit längerem überfällig es in die bestehende Managementapplikation zu integrieren.

Dank der übermässig langen Testphase des Prototypen sind nun die Anforderungen an das definitive Tool gut bekannt. Daher soll eine Webapplikation mit Django erstellt werden, mit welcher die Geschäftsleitung von allink.creative die Ressourcenplanung der Mitarbeiter vornehmen kann. Damit soll eine ältere Webapplikation abgelöst werden.
    
\section{Detaillierte Aufgabenstellung}
    
Das bestehende Tool namens \"Yatplaner\" soll als Modul im bestehenden Management Tool namens \"allink.planer\" reimplementiert werden. Dabei soll die Bedienbarkeit verbessert werden. Das Ziel ist es das neue Tool so intuitiv bedienen zu können, dass für die Geschäftsleitung keine Schulung nötig ist. Der Praxistest wird voraussichtlich in der letzten IPA Woche an der Wochenplansitzung durchgeführt.

Nebst der begleitenden IPA Dokumentation, wo unter anderem der Funktionsumfang des bestehenden Tools analysiert wird, wird keine zusätzliche Dokumentation gefordert. Der Funktionsumfang des bestehenden Tools soll vom Lernenden in einer Analysephase aufgenommen werden. Dabei sollen die bestehenden Features als Muss- und mögliche neue Features als Kann-Ziele ausformuliert werden.

Der Quellcode des bestehenden Tools ist unter folgender Adresse einsehbar: \\https://github.com/sspross/yatplaner/tree/rails

\section{Mittel und Methoden}
Folgende Technologien sind zwingend zu verwenden:

\begin{itemize}
    \item  Python 2.6
    \item  Django 1.3
    \item  Piston 2.3
    \item  jQuery 1.8
    \item  HTML5
    \item  CSS3
\end{itemize}

Das Tool soll in folgenden Browsern fehlerfrei funktionieren:

\begin{itemize}
    \item Firefox >= 8
    \item Chrome >= 10
    \item Safari >= 5
\end{itemize}

Der Internet Explorer muss explizit nicht unterstützt werden.
    
\section{Vorkenntnisse}
Dem Lernenden sind alle genannten Technologien bereits bekannt. Seit Beginn seines Praktikums im August 2011 setzt er sich damit auseinander. Gewisse Kombinationen wie z.B. mit jQuery einen AJAX Request zu erstellen, sind jedoch Neuland. 
    
\section{Vorarbeiten}
Es findet keine explizite Vorarbeit statt. 
    
\section{Neue Lerninhalte}
Wie bereits erwähnt sind dem Lernenden alle Technologien bereits bekannt. Jedoch sind gewisse Kombinationen noch nie vom Lernenden selbst angewandt worden. Das Know-how ist bei allink ausreichend vorhanden. Der Lernende kann zudem auf eine Vielzahl von bestehenden Projekten zurückgreifen, wo er unzählige Beispiele studieren kann. 
    
\section{Arbeiten in den letzten 6 Monaten}
Der Lernende hat überwiegend Webseiten mit den oben genannten Technologien umgesetzt. Zu seinen umfassendsten Arbeiten zählen bis jetzt eine Webseite eines Immobilienunternehmens und einer Eventplattform eines Finanzkonzerns. Für ersteres arbeitete der Lernende rund 250 Stunden daran. 
